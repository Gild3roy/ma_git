\chapter{Related Work}\label{chap:relatedwork}

From the introduction in the previous chapter it is clear that the expansion of the electric grid is often necessary to cope with the growing demand for electricity in context of the energy transition. In the literature there already exist different approaches for power distribution planning \cite{review}. They vary strongly in the methods they use, from exact numeric methods like non-linear- or dynamic-programming \cite{non_linear_programming, dynamic_programming} to heuristic methods like genetic algorithms \cite{genetic_algo}, tabu search \cite{tabu_search} and ant-colony-optimization. In addition to the different methods, they also differ in terms of side constraints which have to be fulfilled, as well as assumptions about the nature of the grid and inputs which need to be provided. Numeric methods are attractive because they produce exact and optimal solutions. Unfortunately, their performance is often times not good enough for most practical applications. Many challenges of grid planning involve NP-complete problems (for example related to the traveling salesman problem) which lead to very long computational time of numeric methods even for strong computers. To reduce the run-time they need to simplify the problem to a great extend such that the results can not easily be translated to real world applications anymore. Due to this reason, heuristic methods seem to be more promising since they can find acceptable solutions even for NP-complete problems in polynomial time. This speed advantage makes them also applicable to more realistic scenarios without oversimplification. Unfortunately, heuristics can also introduce inaccuracies, which could lead to overconfidence in the produced solution. A thorough testing of the algorithm and an evaluation of the results is therefore necessary. This work uses a heuristic ant colony optimization algorithm for network planning since they showed promising results in the past \cite{ant_system}\cite{bonabeau1999swarm}\cite{ant_coloy_system}. Four sources were especially influential for the creation of this work and are presented in more detail in the upcoming sections. Afterwards, a brief overview of the new contributions to grid planning of this work is given.

\section{Dorigo et. al.}
Ant colony optimization methods are known to perform well on many grid planning problems. They were first introduced by Dorigo et. al. in 1996 in an algorithm called \textit{Ant System} \cite{ant_system}. \textit{Ant System} was tested and evaluated on the famous NP-complete traveling salesman problem. Inspired by real ants astonishing ability of swarm intelligence to find short routes to sources of food Dorigo et. al. used artificial ants which can communicate and learn through pheromone trails left behind by ants of previous iterations.
The basic idea is that shorter paths to the food source will accumulate a higher density of pheromones, which leads to a convergence of most ants towards the shorter path. The artificial ants construct a technically valid solution and in a global pheromone update step the best solution is rewarded by an increase in pheromones whereas the other solutions are punished by a loss of pheromones.

One year later in 1997 Dorigo et. al. published an improved version called \textit{ant colony system} \cite{ant_coloy_system} which uses a local pheromone update in addition to the global pheromone update already used in the original algorithm. This local update removes pheromone on the paths already used by the ant, such that the next ant is more likely to chose a different path. This leads to a larger part of the search space being explored and reduces that chance of convergence towards a local optimum. The same procedure is also used in the algorithm of this work. A more detailed introduction to ant colony optimization algorithms with examples can be found in Chapter\ref{chap:background}.\\

\section{Gebhard}
In 2021 Gebhard developed an algorithm in his thesis that uses ant colony optimization for expansion planning of low voltage grids called \textit{AntPower} \cite{gebhard2021expansion}. \textit{AntPower} combines conventional power line expansion with a reconfiguration of the power switches. Gebhard provides a great introduction into the planning of grids and the technical foundations for scholars outside electrical engineering like mathematics or computer science. Additionally, he gives a detailed description into how ACO is implemented in the context of grid planning. To evaluate the algorithm real data of a village in rural Germany is used. His results show, that the expansion plan generated by \textit{AntPower} is 60\% cheaper than an expansion plan obtained through conventional, manual planning based on expert knowledge and 64\% cheaper than the expansion plan generated using a local search algorithm.

\section{Rotering}
For the optimization of medium voltage grids via ACO, the dissertation of Rotering \cite{rotering2013zielnetzplanung} gives a great overview.
Rotering developed a procedure for medium voltage target planning using ant colony optimization which is also able to consider controllable loads and generators. In addition to network costs he also conducted an economic evaluation, which is part of the network design, but methodologically separate from the energy supply task of the network.\\
Rotering translates the electric grid into a graph and points out similarities between the medium voltage network problem and other optimization problems like the traveling salesman problem (TSP). Since ant colony algorithms were specifically developed for the TSP Rotering opted to use them as an optimization tool. A crucial step in his procedure is the reduction of the potential connections between local network stations. He only considers edges of triangles created by the Delaunay triangulation as potential lines between networks stations. This dramatically reduces the search space and therefore reduces the runtime of the algorithm (further details in \ref{triangulation}). In addition, a triangle in itself already constitutes a ring structure which is important for the topological constraints of a MV grid (n-1 criterion). Like Dorigo et. al. presented in \cite{ant_coloy_system}, Rotering uses a local and a global pheromone update for his algorithm. After the construction of a legal network design Rotering uses a few optimization methods to further improve the solution. He evaluates the his procedure on two MV grids and receives promising results.

%TODO: get the book in ISE to check sources! (triangulation 155, aco für ms: 150-152) 



\section{Zeller}
Zellers thesis about the planning of medium voltage grids via ACO serves as the basis for this work. The presented ACO algorithm Ant-Power-Medium-Voltage (APMV) is s a further development of the algorithm Zeller presents in \cite{zeller2021planung}. Zeller already uses some concepts like the Delaunay-Triangulation, a local and global pheromone update and pheromone evaporation in his algorithm. Zeller also considered special cases originating from the triangulation which could lead to faulty networks (called "back-and-forth" cases). They are explained in further detail in \ref{backandforth}. In these cases Zeller extended and optimized the procedure to build valid solutions. Besides a practical analysis and a parameter study he conducted a theoretical analysis and provided several proofs to further legitimize the procedure. He showed, that given enough runtime the algorithm would always find the global optimum and that all buses would be connected. Additionally Zeller showed that the topological constraints would always be fulfilled. For evaluation of the developed algorithm Zeller used manually built synthetic grids with sizes of 5 to 30 nodes.

\newpage
\section{New contributions}
This work contributes to grid planning via ACO in the following ways:
\begin{enumerate}
	\setlength\itemsep{-0.5em}
	\item An improved algorithm for optimizing medium voltage grids called \textit{Ant-Power-Medium-Voltage} (\textit{APMV}) is presented.
	\item The algorithm is tested and evaluated on a real world cross-voltage grid.
	\item Cases, which previously lead to faulty network structures are intercepted.
	\item A parameter study to find the algorithms best performance was conducted.
\end{enumerate}

Regarding grid planning, so far in the literature many methods for grid planning are provided without testing and evaluating them on real world examples. Scientific research however, is only useful if it can be converted into an improvement of the practical application. Therefore, the aim of this work is to use the algorithm to optimize a real world cross-voltage grid and evaluate its performance. This grid is taken from a village which is located in rural Germany and and resembles a typical application example.\\
Besides applying existing research on a real world example, this work also presents an ACO algorithm with several improvements. Firstly, the accuracy of the cost function is increased by several measures. Secondly, the laying of new lines is done alongside streets to make more realistic estimations. Thirdly, overlapping rings in the network are handled in a sensible way. And lastly, cases which previously lead to faulty network structures are intercepted.





%to combine different methods to gain better overall results. Thus, the algorithm uses information from the low voltage level about potentially benefiting transformer modifications. It could be useful from the perspective of the low voltage level to relocate, add or delete a transformer to reduce costs or load flow violations. This method has been developed and evaluated by Verheggen \cite{verheggen2016kombinierte} and implemented by John in \cite{robert_john}. However, restructuring the transformers has an influence on the middle voltage level and therefore gains for the low voltage level have to be weighed out with potential losses for middle voltage level. \\






