\chapter{Introduction}\label{chap:introduction}

The Climate crisis requires a rapid transition from fossil to renewable energy sources to reduce carbon emissions. In comparison to conventional power plants, renewable energy resources like wind and solar are placed more decentralized across the electric grid and produce energy in less predictable patterns. Additionally, the demand for electricity will increase drastically in the upcoming years notably due to the widespread usage of heat pumps and electric vehicles \cite{brutto_stromverbrauch}. This poses new challenges on the grid to stably deliver electricity from the producer to the consumer. To cope with the additional requirements, the power grid must be expanded in many places. According to a report filed by the federal network agency of Germany in 2021, the cost of the network expansion for the coming ten years amounts to 15.84 billion Euros, of which 7.86 billion Euros are estimated for middle voltage (MV) grid expansion \cite{Ausbaubericht_2021_page14}. The development of tools for smart MV grid expansion is therefore not only relevant for the future supply of electricity but also for public and private funding. \\
However, the planning of distribution grid expansion gets increasingly complex, due to the new requirements mentioned above. This makes is it more difficult for experts to rely on manual planning only. Instead, distribution system operators increasingly use automated planning methods for assistance.
\cite{rotering2013zielnetzplanung} categorizes network planning into \textit{target network planning} and \textit{expansion planning}, depending on the applied time horizon. \textit{Target network planning} has a scope of a longer time horizon (more than ten years) and does not take existing grid infrastructure into account. It assumes that a new grid can be built from scratch and tries to find an optimal solution which fulfills all given requirements. This is typically used for the planning of developing areas and long-term grid transformations. \textit{Network expansion planning} on the other hand takes the existing grid infrastructure into account and tries to modify it in a way that it can cope with the short- and mid-term demands (time horizon five to ten years). Generally, \textit{target planning} is done in a first step and afterwards the existing network is modified using \textit{expansion planning} in the direction of the target network solution. In this work network planning generally refers to \textit{target network planning}.\\
Fo transmission efficiency reasons the grid is separated into a high-, medium- and low-voltage level. Each of them with different specific requirements which an automated planning tool must consider. A special characteristic of the medium-voltage level is, for example, the \textit{n-1 criterion}, according to which the supply to each station must be guaranteed despite the failure of one element. It therefore stands to reason that an automated planning tool would be developed for a specific voltage level. The presented work addresses the problem of medium voltage grid planning via Ant Colony Optimization (ACO). ACO is a heuristic optimization method which performed well on grid planning problems in the past \cite{rotering2013zielnetzplanung}.

This work is structured in the following way. Chapter \ref{chap:relatedwork} gives an overview of the research in the field. Chapter \ref{chap:background} provides background information on electric grids, an introduction to ant colony optimization and how it can be used in the context of grid planning. Chapter \ref{chap:approach} formulates the optimization problem and follows with the presentation of \textit{Ant-Power-Medium-Voltage (APMV)}, an algorithm for the planning of MV grids. Subsequently, the algorithm is tested and evaluated on a real-world example grid in Chapter \ref{chap:experiments}. Chapter \ref{chap:conclusion} concludes this work with an outlook on future research.